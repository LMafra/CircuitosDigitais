	\PassOptionsToPackage{brazil,american}{babel}
\documentclass[12pt]{article}

\usepackage{sbc-template}
\usepackage[brazil,american]{babel}
\usepackage[utf8]{inputenc}

\usepackage{graphicx}
\usepackage{url}
\usepackage{float}
\usepackage{listings}	
\usepackage{color}
\usepackage{todonotes}
\usepackage{algorithmic}
\usepackage{algorithm}
\usepackage{hyperref}

\sloppy

\title{Experimento 10\\ 
	CONTADOR ASSÍNCRONO}

\author{
	Lucas Mafra Chagas, 12/0126443 \\
	Marcelo Giordano Martins Costa de Oliveira,  12/0037301
}


\address{Dep. Ciência da Computação -- Universidade de Brasília (UnB)\\
	CiC 116351 - Circuistos Digitais - Turma C
	\email{\{giordano.marcelo, chagas.lucas.mafra\}@gmail.com}
}

\begin{document}
	
	\maketitle
	
	\begin{abstract}
	
	\end{abstract}
	
	\begin{resumo} 
	
	\end{resumo}
	
	\section{Objetivos}
	\label{sec:Objetivos}
		Montar um contador assíncrono binário progressivo de 4 estágios, com flip-flops
		JK.Verificar a ocorrência de estados transitórios. Comparar com o funcionamento de
		um contador síncrono em anel. Projetar e montar um contador assíncrono binário
		reversível de 4 estágios, com flip-flops JK.
	
	
	\section{Materiais} 
	\label{sec:Materiais}
	
	\begin{itemize}
		\item Painel Digital
		
	\end{itemize}
	
	\section{Introdução}
	\label{sec:Introducao}
	
	
	
	\section{Procedimentos}
	\label{sec:Procedimentos}
	

	
	\subsection{Montar um flip-flop “T” utilizando apenas portas NAND.}

	
	\subsection{Montar um flip-flop “D” gatilhado por nível.}
	\label{2.2}
	

	
	\subsection{Montar um flip-flop “D” gatilhado pela borda.}
	\label{2.3}
	

	\subsection{Montar, verificar e explicar o funcionamento de um flip-flop “D” com circuito auxiliar de produção de pulsos.}
	\label{2.4}
	

	
	\subsection{Projetar um circuito sequencial que detecta o sentido de movimento de veículos em uma rua.}
	\label{2.5}
	
	 
	
	
	
	\section{Análise dos Resultados}
	\label{sec:Resultados}
	
	\subsection{Montar um flip-flop “T” utilizando apenas portas NAND.}

	
	\subsection{Montar um flip-flop “D” gatilhado por nível.}
	
	
	\subsection{Montar um flip-flop “D” gatilhado pela borda.}
	

	
	
	\subsection{Montar, verificar e explicar o funcionamento de um flip-flop “D” com circuito auxiliar de produção de pulsos.}

	
	
	\section{Conclusão}
	\label{sec:Conclusao}
	

	
	\newpage 
	% Colocar aqui apenas as respostas dos itens da Auto-Avaliação
	\section*{Auto-Avaliação}
	
	\begin{enumerate}
		\item B
		\item C
		\item A
		\item C
		\item C
		\item E
		\item C
		\item B
		\item B
		\item A
	\end{enumerate}
	
	
\end{document}