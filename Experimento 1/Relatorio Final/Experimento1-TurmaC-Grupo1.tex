%%%%%%%%%%%%%%%%%%%%%%%%%%%%%%%%%%%%%%%%%%%%%%%%%%%%%%%%
% Este é um documento que servirá de modelo para
% os relatórios feitos na disciplina Circuitos Digitais
% 2016-2
%%%%%%%%%%%%%%%%%%%%%%%%%%%%%%%%%%%%%%%%%%%%%%%%%%%%%%%%%

\documentclass[12pt]{article}

\usepackage{sbc-template}
\usepackage[brazil,american]{babel}
\usepackage[utf8]{inputenc}

\usepackage{graphicx}
\usepackage{url}
\usepackage{float}
\usepackage{listings}
\usepackage{color}
\usepackage{todonotes}
\usepackage{algorithmic}
\usepackage{algorithm}
\usepackage{hyperref}
     
\sloppy

\title{Experimento 1\\ 
Portas lógicas AND, OR e NOT}

\author{Lucas Mafra Chagas, 12/0126443\\
        Marcelo Giordano Martins Costa de Oliveira,  12/0037301\\
}


\address{Dep. Ciência da Computação -- Universidade de Brasília (UnB)\\
  CiC 116351 - Circuitos Digitais - Turma A
  \email{\{giordano.marcelo, chagas.lucas.mafra\}@gmail.com}
}

\begin{document} 

\maketitle

 \begin{abstract}
   This essay has the intention to give the student the first contact with de digital panel and logic gates.
 \end{abstract}
     
 \begin{resumo} 
  O intuito deste experimento é dar o primeiro contato com o painel digital e a utilização de portas lógicas.
 \end{resumo}


\section{Objetivos}
\label{sec:Objetivos}

Fornecer ao aluno um contato inicial com o painel. São apresentadas as portas AND, OR e
NOT e os conceitos de atraso em portas lógicas e nível de ruído em circuitos digitais.

\section{Materiais} 
\label{sec:Materiais}

\begin{itemize}
    \item Painel Digital;
    
    \item \textit{protoboard};
    
    \item Fios conectores;
    
    \item Portas lógicas AND (7408), OR(7432) e NOT(7404);
    
    \item Fios conectores;
    
    \item Multímetro;
    
    \item Ponta lógica;
    
\end{itemize}


\section{Introdução}
\label{sec:Introducao}


\section{Procedimentos}
\label{sec:Procedimentos}

Este experimento é dividido em quatro etapas.

\begin{itemize}
	\item A primeira etapa tem o foco de testar o funcionamento das portas lógicas AND e OR. Para isso, o aluno precisa verificar se as entradas da placa condizem com a saída esperada, além de verificar o valor da tensão nas 4 saídas dos chips.
	\item Na segunda etapa, o aluno precisa projetar e implementar uma porta OR usando apenas portas AND e NOT. Para realização desta etapa, o estudante precisa verificar se o desenho projetado realiza a função desejada no final.
	\item Na terceira etapa, assim como na segunda, o aluno precisa projetar e implementar uma porta AND usando apenas portas OR e NOT. Como realizado na segunda etapa, o aluno precisa verificar se o desenho projetado realiza a função desejada no final.
	\item Na quarta etapa, o intuito é verificar a existência dos atrasos de propagação em portas. Para isso, o aluno precisa fazer com que a haja uma saída da placa, onde, em série, uma vai direto para a porta AND e a outra é negada cinco vezes e enviada para a porta AND. É verificado se surgiu algum pulso com ajuda da ponta lógica. 
\end{itemize}


\subsection{Funcionamento das portas lógicas AND e OR}
\label{sec:Porta Lógica}


\begin{table}
	\centering
	\begin{tabular}{|c|c|c|c|c|c|c|c|}
	\cline{1-6}
	\multicolumn{1}{|c|}{A} & \multicolumn{1}{|c|}{B} & \multicolumn{1}{|c|}{S1} & \multicolumn{1}{|c|}{S2} & \multicolumn{1}{|c|}{S3} & \multicolumn{1}{|c|}{S4}\\
	\hline
	0 & 0 & 0.01 & 0.02 & 0.02 & 0.02\\
	0 & 1 & 0.01 & 0.02 & 0.02 & 0.02\\
	1 & 0 & 0.01 & 0.02 & 0.02 & 0.02\\
	1 & 1 & 4.99 & 4.98 & 4.96 & 4.97\\
	\hline
	\end{tabular}
	\label{Porta AND}
\end{table}

\begin{table}
	\centering
	\begin{tabular}{|c|c|c|c|c|c|c|c|}
	\cline{1-6}
	\multicolumn{1}{|c|}{A} & \multicolumn{1}{|c|}{B} & \multicolumn{1}{|c|}{S1} & \multicolumn{1}{|c|}{S2} & \multicolumn{1}{|c|}{S3} & \multicolumn{1}{|c|}{S4}\\
	\hline
	0 & 0 & 0.03 & 0.01 & 0.01 & 0.05\\
	0 & 1 & 4.98 & 4.97 & 4.98 & 4.98\\
	1 & 0 & 4.98 & 4.97 & 4.97 & 4.98\\
	1 & 1 & 4.98 & 4.97 & 4.97 & 4.97\\
	\hline
	\end{tabular}
	\label{Porta OR}
\end{table}


\subsection{Implementação da porta OR com portas AND e NOT}
\label{sec:NOTAND}

\begin{figure}[H]
\centering
\includegraphics[width=.5\textwidth]{Porta_OR.jpeg}
\caption{Uma figura}
\label{fig:portaor}
\end{figure}

A Figura~\ref{fig:portaor} apresenta um exemplo de como usar e citar uma figura.

\begin{table}
	\centering
	\begin{tabular}{|c|c|c|}
	\cline{1-3}
	\multicolumn{1}{|c|}{A} & \multicolumn{1}{|c|}{B} & \multicolumn{1}{|c|}{S1}\\
	\hline
	0 & 0 & 0\\
	0 & 1 & 1\\
	1 & 0 & 1\\
	1 & 1 & 1\\
	\hline
	\end{tabular}
	\label{ANDNOTOR}
\end{table}

\subsection{Implementação da porta AND com portas OR e NOT}
\label{sec:NOTOR}

\begin{figure}[H]
\centering
\includegraphics[width=.5\textwidth]{Porta_AND.jpeg}
\caption{Uma figura}
\label{fig:portaand}
\end{figure}

A Figura~\ref{fig:portaand} apresenta um exemplo de como usar e citar uma figura.

\begin{table}
	\centering
	\begin{tabular}{|c|c|c|}
	\cline{1-3}
	\multicolumn{1}{|c|}{A} & \multicolumn{1}{|c|}{B} & \multicolumn{1}{|c|}{S1}\\
	\hline
	0 & 0 & 0\\
	0 & 1 & 0\\
	1 & 0 & 0\\
	1 & 1 & 1\\
	\hline
	\end{tabular}
	\label{ORNOTAND}
\end{table}

\subsection{Atraso de Propagação em Portas}
\label{sec:atraso}

Aqui temos um exemplo de como criar um hiperlink. Veja
\href{https://www.youtube.com/watch?v=EcNxjxKRQ6E}{aqui} um exemplo de vídeo.

Sempre identifique no site do vídeo:
\begin{itemize}
    \item o experimento: Experimento 7;
    \item semestre: 2016-2;
    \item a disciplina: CiC 116351 - Circuitos Digitais - Turma B;
    \item a universidade: Universidade de Brasília (UnB);
    \item os nomes dos componentes do grupo.
\end{itemize}

\section{Análise dos Resultados}
\label{sec:Resultados}


\section{Conclusão}
\label{sec:Conclusao}


\bibliographystyle{sbc}
\bibliography{relatorio}


\newpage 
% Colocar aqui apenas as respostas dos itens da Auto-Avaliação
\section*{Auto-Avaliação}

\begin{enumerate}
    \item a
    \item c
    \item b
    \item d
\end{enumerate}


\end{document}
