%%%%%%%%%%%%%%%%%%%%%%%%%%%%%%%%%%%%%%%%%%%%%%%%%%%%%%%%
% Este é um documento que servirá de modelo para
% os relatórios feitos na disciplina Circuitos Digitais
% 2016-2
%%%%%%%%%%%%%%%%%%%%%%%%%%%%%%%%%%%%%%%%%%%%%%%%%%%%%%%%%

\PassOptionsToPackage{brazil,american}{babel}
\documentclass[12pt]{article}

\usepackage{sbc-template}
\usepackage[brazil,american]{babel}
\usepackage[utf8]{inputenc}

\usepackage{graphicx}
\usepackage{url}
\usepackage{float}
\usepackage{listings}
\usepackage{color}
\usepackage{todonotes}
\usepackage{algorithmic}
\usepackage{algorithm}
\usepackage{hyperref}
     
\sloppy

\title{Experimento 7\\ 
Circuitos Combinacionais: Multiplexadores}

\author{Lucas Mafra Chagas, 12/0126443\\
        Marcelo Giordano Martins Costa de Oliveira,  12/0037301\\
}


\address{Dep. Ciência da Computação -- Universidade de Brasília (UnB)\\
  CiC 116351 - Circuitos Digitais - Turma A
  \email{\{giordano.marcelo, chagas.lucas.mafra\}@gmail.com}
}

\begin{document} 

\maketitle

 \begin{abstract}
   Write here a short summary of the report in English. This corresponds to the Experiment 7 report on combinational circuits, specifically the multiplexers.
 \end{abstract}
     
 \begin{resumo} 
  Escreva aqui um pequeno resumo do relatório. Este corresponde ao relatório do Experimento 7 sobre circuitos combinacionais, especificamente os multiplexadores.
 \end{resumo}


\section{Objetivos}
\label{sec:Objetivos}

Fornecer ao aluno um contato inicial com o painel. São apresentadas as portas AND, OR e
NOT e os conceitos de atraso em portas lógicas e nível de ruído em circuitos digitais.

\section{Materiais} 
\label{sec:Materiais}

\begin{itemize}
    \item Painel Digital;
    
    \item \textit{protoboard};
    
    \item Fios conectores;
    
    \item Portas lógicas AND (7408), OR(7432) e NOT(7404);
    
    \item Fios conectores;
    
    \item Multímetro;
    
    \item Ponta lógica;
    
\end{itemize}


\section{Introdução}
\label{sec:Introducao}

Os sistemas digitais que conhecemos tem como base o uso de operações lógicas que terão sempre apenas duas respostas possíveis: verdadeiro (representado também pelo valor 1) e falso (representado também pelo valor 0). Os circuitos digitais permitem que essas operações lógicas sejam implementadas, consequentemente permitindo que as máquinas realizem as operações que exigimos delas. 
Nos circuitos, os valores ‘verdadeiro’ e ‘falso’ interpretados pelo sistema digital são gerados a partir de tensões. Na lógica positiva, temos que um nível alto de tensão é associado ao nível lógico 1, enquanto o nível baixo de tensão é associado ao nível lógico 0.
Para este experimento, o circuitos integrados que foram utilizados pertencem à família TTL (Transistor-Transistor-Logic). Para estes circuitos temos uma alimentação de 5,0 Volts, onde o nível lógico 1 terá saída de 5,0 Volts(o maior valor) e o nível lógico 0 terá uma saída de 0 Volts. É claro que isso não é um valor exato. Para valores de entrada, temos que o circuito reconhece como 0 valores até 0,8 volts, e como 1 valores de 2,0 a 5,5 Volts. Isso é necessário pois experimentalmente precisamos de um intervalo de erros na hora de realizar leituras. Na saída, o circuito TTL representa o nível lógico 0 no intervalo de 0 a 0,4 Volts e o nível lógico 1 no intervalo de 2,4 a 5,0 Volts.
Para realizar as operações lógicas com os níveis lógicos é necessária a utilização de portas lógicas. As operações lógicas básicas são as operações AND (A e B), OR (A ou B) e NOT (não A). Já nos circuitos, temos 2 pinos que destinam-se à alimentação do circuito e 12 que se destinam às portas, que geralmente tem 2 entradas e uma saída, gerando um total de 4 portas por circuito integrado. Combinando essas portas é possível gerar diversas operações lógicas com diferentes níveis de complexidade. As portas lógicas recebem informações em nível lógico e geram uma saída também em nível lógico. Portanto, precisamos apenas de dois valores qualquer que seja a operação a ser feita.
Neste experimento, trabalharemos com estas portas lógicas, estudaremos as tabelas verdade geradas para cada combinação de valores em cada porta e mostraremos que apesar de eficiente, ao lidarmos com portas lógicas temos que considerar que cada uma delas possui um tempo para processar uma informação, provando que erros são presentes até nos sistemas mais simples.

\section{Procedimentos}
\label{sec:Procedimentos}

Escreva nesta seção os diversos itens pedidos no experimentos. 

\subsection{Funcionamento das portas lógicas AND e OR}
\label{sec:Porta Lógica}


\begin{table}
	\centering
	\begin{tabular}{|c|c|c|c|c|c|c|c|}
	\cline{1-6}
	\multicolumn{1}{|c|}{A} & \multicolumn{1}{|c|}{B} & \multicolumn{1}{|c|}{S1} & \multicolumn{1}{|c|}{S2} & \multicolumn{1}{|c|}{S3} & \multicolumn{1}{|c|}{S4}\\
	\hline
	0 & 0 & 0.01 & 0.02 & 0.02 & 0.02\\
	0 & 1 & 0.01 & 0.02 & 0.02 & 0.02\\
	1 & 0 & 0.01 & 0.02 & 0.02 & 0.02\\
	1 & 1 & 4.99 & 4.98 & 4.96 & 4.97\\
	\hline
	\end{tabular}
	\label{Porta AND}
\end{table}

\begin{table}
	\centering
	\begin{tabular}{|c|c|c|c|c|c|c|c|}
	\cline{1-6}
	\multicolumn{1}{|c|}{A} & \multicolumn{1}{|c|}{B} & \multicolumn{1}{|c|}{S1} & \multicolumn{1}{|c|}{S2} & \multicolumn{1}{|c|}{S3} & \multicolumn{1}{|c|}{S4}\\
	\hline
	0 & 0 & 0.03 & 0.01 & 0.01 & 0.05\\
	0 & 1 & 4.98 & 4.97 & 4.98 & 4.98\\
	1 & 0 & 4.98 & 4.97 & 4.97 & 4.98\\
	1 & 1 & 4.98 & 4.97 & 4.97 & 4.97\\
	\hline
	\end{tabular}
	\label{Porta OR}
\end{table}

Descrever o experimento realizado. Sempre  que colocar uma figura deve-se explicar o que se pretende que o leitor veja, ou uma análise logo após a figura. 


Aqui temos um exemplo de como citar uma URL na bibliografia~\cite{systemverilog}.


É apresentado acima como fazer uma listagem não numerada.

\subsection{Implementação da porta OR com portas AND e NOT}
\label{sec:NOTAND}

\begin{figure}[H]
\centering
\includegraphics[width=.5\textwidth]{Porta_OR.jpeg}
\caption{Uma figura}
\label{fig:portaor}
\end{figure}

A Figura~\ref{fig:portaor} apresenta um exemplo de como usar e citar uma figura.

\begin{table}
	\centering
	\begin{tabular}{|c|c|c|}
	\cline{1-3}
	\multicolumn{1}{|c|}{A} & \multicolumn{1}{|c|}{B} & \multicolumn{1}{|c|}{S1}\\
	\hline
	0 & 0 & 0\\
	0 & 1 & 1\\
	1 & 0 & 1\\
	1 & 1 & 1\\
	\hline
	\end{tabular}
	\label{ANDNOTOR}
\end{table}

\subsection{Implementação da porta AND com portas OR e NOT}
\label{sec:NOTOR}

\begin{figure}[H]
\centering
\includegraphics[width=.5\textwidth]{Porta_AND.jpeg}
\caption{Uma figura}
\label{fig:portaand}
\end{figure}

A Figura~\ref{fig:portaand} apresenta um exemplo de como usar e citar uma figura.

\begin{table}
	\centering
	\begin{tabular}{|c|c|c|}
	\cline{1-3}
	\multicolumn{1}{|c|}{A} & \multicolumn{1}{|c|}{B} & \multicolumn{1}{|c|}{S1}\\
	\hline
	0 & 0 & 0\\
	0 & 1 & 0\\
	1 & 0 & 0\\
	1 & 1 & 1\\
	\hline
	\end{tabular}
	\label{ORNOTAND}
\end{table}

\subsection{Atraso de Propagação em Portas}
\label{sec:atraso}

Aqui temos um exemplo de como criar um hiperlink. Veja
\href{https://www.youtube.com/watch?v=EcNxjxKRQ6E}{aqui} um exemplo de vídeo.

Sempre identifique no site do vídeo:
\begin{itemize}
    \item o experimento: Experimento 7;
    \item semestre: 2016-2;
    \item a disciplina: CiC 116351 - Circuitos Digitais - Turma B;
    \item a universidade: Universidade de Brasília (UnB);
    \item os nomes dos componentes do grupo.
\end{itemize}

\section{Análise dos Resultados}
\label{sec:Resultados}

Faça uma análise crítica dos resultados obtidos nos experimentos. Esta análise pode ser feita item a item ou de uma forma geral.

Dica: Use pesquisa na Internet para tirar as dúvidas sobre edição em \LaTeX .

\section{Conclusão}
\label{sec:Conclusao}

Concluir o relatório explanando rapidamente o que foi feito e os resultados obtidos, sempre correlacionando com os objetivos do experimento apresentado na Seção~\ref{sec:Objetivos}. 


\bibliographystyle{sbc}
\bibliography{relatorio}


\newpage 
% Colocar aqui apenas as respostas dos itens da Auto-Avaliação
\section*{Auto-Avaliação}

\begin{enumerate}
    \item a
    \item c
    \item b
    \item d
\end{enumerate}


\end{document}
